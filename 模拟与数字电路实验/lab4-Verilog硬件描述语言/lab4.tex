\documentclass[12pt]{ctexart}
\usepackage{graphicx}
\usepackage{fancyhdr}
\usepackage{amsmath}
\usepackage{geometry}
\geometry{papersize={21cm,29.7cm}}
\geometry{left=2cm,right=2cm,top=3cm,bottom=2cm}
\pagestyle{fancy}
\lhead{姓名:侯超群}
\chead{\today}
\rhead{学号:PB21111618}
\lfoot{}
\cfoot{\thepage}
\rfoot{}
\renewcommand{\headrulewidth}{0.4pt}
\renewcommand{\headwidth}{\textwidth}
\renewcommand{\footrulewidth}{0pt}
\begin{document}
\section*{模拟与数字电路实验报告}
\noindent{\bf1.实验题目:}实验04 Verilog硬件描述语言\\
{\bf2.实验目的:}掌握 Verilog HDL 常用语法,能够熟练阅读并理解 Verilog 代码,能够设计较复杂的数字功能电路,能够将 Verilog 代码与实际硬件相对应\\
{\bf3.实验环境:}CPU:AMD Ryzen 5 5600H with Radeon Graphics 3.30GHz;内存:16GB;操作系统:Windows 10;软件平台:Logisim;\\
{\bf4.实验练习:}\\
{\bf1).problem1:}由于ifelse语句一般出现在always语句的过程语句部分,而不能在模块内部单独出现,故用always@(*)语句实现组合逻辑;
同时always语句内部被赋值的信号应被定义为reg类型,故模块端口定义部分定义reg b;\\
module test(\\
input a,\\
output reg b\\
);\\
always@(*)\\
begin\\
\hspace*{0.5cm}if(a)\\
\hspace*{1cm}b = 1'b0;\\
\hspace*{0.5cm}else\\
\hspace*{1cm}b = 1'b1;\\
end\\
endmodule\\
{\bf2).problem2:}由于always语句内实现对b的赋值,可知b与a有相同的位宽,且被定义为reg类型;末尾有endmodule语句结束;\\
module test(\\
input [4:0] a,\\
output reg [4:0] b\\
);\\
always@(*)\\
\hspace*{0.5cm}b = a;\\
endmodule\\
{\bf3).problem3:}各输出信号的值如下:\\
module test(\\
input [7:0] a, b,//a = 8'b0011\_0011,  b = 8'b1111\_0000\\
output [7:0] c, d, e, f, g, h, i, j, k\\
);\\
\hspace*{0.5cm}assign c = a \& b;       //按位与8'b0011\_0000\\
\hspace*{0.5cm}assign d = a | b;       //按位或8'b1111\_0011\\
\hspace*{0.5cm}assign e = a $\wedge$ b;       //按位异或8'b1100\_0011\\
\hspace*{0.5cm}assign f = $\sim$ a;          //按位取反8'b1100\_1100\\
\hspace*{0.5cm}assign g = $\left \{  a[3:0], b[3:0]\right \}$ ;    //拼接操作符8'b0011\_0000\\
\hspace*{0.5cm}assign h = a >> 3;              //右移8'b0000\_0110\\
\hspace*{0.5cm}assign i = \&b;                  //归约与8'bxxxx\_ xxx0\\
\hspace*{0.5cm}assign j = (a > b) ? a : b;     //条件操作符8'b1111\_0000\\
\hspace*{0.5cm}assign k = a - b;               //算术运算符8'b0100\_0011\\
endmodule\\
{\bf4).problem4:}assign语句将逻辑表达式的值赋给wire类型信号,因此c信号的类型应为wire默认类型;
因此在模块调用时,temp信号的类型同样应设为wire,在模块关联中应采取相同的关联方式,故如下保证一致性;\\
module sub\_test(\\
input a, b,\\
output  c\\
);\\
\hspace*{0.5cm}assign c = (a < b) ? a : b;\\
endmodule\\\\
module test(\\
input a, b, c,\\
output o\\
);\\
\hspace*{0.5cm}wire temp;\\
\hspace*{0.5cm}sub\_test test\_inst1(.a(a), .b(b), .c(temp));\\
\hspace*{0.5cm}sub\_test test\_inst2(temp, c, o);\\
endmodule\\
{\bf5).problem5:}output语句应在端口的定义部分;同时由于模块化实例只需要一次,故不能出现在always语句中;\\
module sub\_test(\\
input a, b,\\
output o\\
);\\
\hspace*{0.5cm}assign o = a + b;\\
endmodule\\\\
module test(\\
input a, b,\\
output c\\
);\\
\hspace*{0.5cm}sub\_test test\_inst(a, b, c);\\
endmodule\\
{\bf5.总结与思考:}\\
{\bf1).收获:}本次实验对于Verilog语法有了较为基础的了解,对模块的基本组成部分基本掌握;\\
{\bf2).评价:}实验作为Verilog语法的基本入门内容,其内容相对而言不是较难,设置合理;\\
{\bf3).建议:}实验内容设置合理,无较大建议;
\end{document}

