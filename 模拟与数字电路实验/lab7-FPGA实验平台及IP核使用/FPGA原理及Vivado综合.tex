\documentclass[12pt]{ctexart}
\usepackage{graphicx}
\usepackage{fancyhdr}
\usepackage{amsmath}
\usepackage{geometry}
\usepackage{xcolor}
\usepackage{listings}


\lstset{
 columns=fixed,       
 numbers=left,                                        % 在左侧显示行号
 numberstyle=\tiny\color{gray},                       % 设定行号格式
 frame=none,                                          % 不显示背景边框
 backgroundcolor=\color[RGB]{245,245,244},            % 设定背景颜色
 keywordstyle=\color[RGB]{40,40,255},                 % 设定关键字颜色
 numberstyle=\footnotesize\color{darkgray},           
 commentstyle=\it\color[RGB]{0,96,96},                % 设置代码注释的格式
 stringstyle=\rmfamily\slshape\color[RGB]{128,0,0},   % 设置字符串格式
 showstringspaces=false,                              % 不显示字符串中的空格
 language=c,                                        % 设置语言
}


\geometry{papersize={21cm,29.7cm}}
\geometry{left=2cm,right=2cm,top=3cm,bottom=2cm}
\pagestyle{fancy}
\lhead{姓名:侯超群}
\chead{\today}
\rhead{学号:PB21111618}
\lfoot{}
\cfoot{\thepage}
\rfoot{}
\renewcommand{\headrulewidth}{0.4pt}
\renewcommand{\headwidth}{\textwidth}
\renewcommand{\footrulewidth}{0pt}
\begin{document}
\section*{模拟与数字电路实验报告}
\noindent{\bf1.实验题目:}实验07 FPGA实验平台及IP核使用 \\
{\bf2.实验目的:}熟悉 FPGAOL 在线实验平台结构及使用,掌握 FPGA 开发各关键环节,学会使用 IP 核(知识产权核)\\
{\bf3.实验平台:}Vivado软件\\
{\bf4.实验练习:}\\
{\bf1).问题一:}对于一个16*8bit的ROM例化并初始化,接到七段数码管上显示与开关相对应的十六进制数字\\
根据数码管对应数字二进制表示存入ROM,例化如下:\\
\includegraphics[width = .9\textwidth]{problem1.jpg}\\
verilog代码如下:
\begin{lstlisting}
  module problem1(
    input [3:0] sw,
    input clk,
    output [7:0]led
      );
  dist_mem_gen_0 dist_mem_gen_0(
    .a  (sw),
    .spo(led)
      );
endmodule
\end{lstlisting}
约束文件如下:
\begin{lstlisting}
set_property  -dict {PACKAGE_PIN E3 IOSTANDARD LVCMOS33 } 
  [get_ports { clk }]; 

set_property -dict {PACKAGE_PIN C17 IOSTANDARD LVCMOS33 } 
  [get_ports { led[0] }];
set_property -dict {PACKAGE_PIN D18 IOSTANDARD LVCMOS33 } 
  [get_ports { led[1] }];
set_property -dict {PACKAGE_PIN E18 IOSTANDARD LVCMOS33 } 
  [get_ports { led[2] }];
set_property -dict {PACKAGE_PIN G17 IOSTANDARD LVCMOS33 } 
  [get_ports { led[3] }];
set_property -dict {PACKAGE_PIN D17 IOSTANDARD LVCMOS33 } 
  [get_ports { led[4] }];
set_property -dict {PACKAGE_PIN E17 IOSTANDARD LVCMOS33 } 
  [get_ports { led[5] }];
set_property -dict {PACKAGE_PIN F18 IOSTANDARD LVCMOS33 } 
  [get_ports { led[6] }];
set_property -dict {PACKAGE_PIN G18 IOSTANDARD LVCMOS33 } 
  [get_ports { led[7] }];

set_property -dict {PACKAGE_PIN D14 IOSTANDARD LVCMOS33 } 
  [get_ports { sw[0] }];
set_property -dict {PACKAGE_PIN F16 IOSTANDARD LVCMOS33 } 
  [get_ports { sw[1] }];
set_property -dict {PACKAGE_PIN G16 IOSTANDARD LVCMOS33 } 
  [get_ports { sw[2] }];
set_property -dict {PACKAGE_PIN H14 IOSTANDARD LVCMOS33 } 
  [get_ports { sw[3] }];
\end{lstlisting}
效果如下:\\
\includegraphics[width = .9\textwidth]{pro.jpg}\\
{\bf2).问题二:}根据8个开关作为输入,十六进制数码管作为输出,采取对数码管扫描的方式,使得在视觉上同时显示在两个数码管上,
由于扫描频率不能过快或者过慢,采取100hz作为扫描频率;\\
verilog代码如下:
\begin{lstlisting}
  module problem2(
    input [7:0]sw,
    input clk,
    output reg [2:0]AN,
    output reg [3:0]D
        );
    reg [15:0] cnt;
    always@(posedge clk)
    begin
        if(cnt >= 10000)
            cnt <= 0;
         else
            cnt <= cnt + 1;
    end
    always@(posedge clk)
    begin
        if(cnt == 0)
        begin
            if(AN == 3'b000)
                AN = 3'b001;
            else    
                AN = 3'b000;
        end 
    end
    always@(posedge clk)
    begin
        if(AN == 3'b000)
            D <= sw[3:0];
        else 
            D <= sw[7:4];
    end
    endmodule
\end{lstlisting}
约束文件如下:
\begin{lstlisting}
set_property  -dict {PACKAGE_PIN E3 IOSTANDARD LVCMOS33 } 
  [get_ports { clk }]; 

set_property -dict {PACKAGE_PIN A14 IOSTANDARD LVCMOS33 } 
  [get_ports { D[0] }];
set_property -dict {PACKAGE_PIN A13 IOSTANDARD LVCMOS33 } 
  [get_ports { D[1] }];
set_property -dict {PACKAGE_PIN A16 IOSTANDARD LVCMOS33 } 
  [get_ports { D[2] }];
set_property -dict {PACKAGE_PIN A15 IOSTANDARD LVCMOS33 } 
  [get_ports { D[3] }];
set_property -dict {PACKAGE_PIN B17 IOSTANDARD LVCMOS33 } 
  [get_ports { AN[0] }];
set_property -dict {PACKAGE_PIN B16 IOSTANDARD LVCMOS33 } 
  [get_ports { AN[1] }];
set_property -dict {PACKAGE_PIN A18 IOSTANDARD LVCMOS33 } 
  [get_ports { AN[2] }];

set_property -dict {PACKAGE_PIN D14 IOSTANDARD LVCMOS33 } 
  [get_ports { sw[0] }];
set_property -dict {PACKAGE_PIN F16 IOSTANDARD LVCMOS33 } 
  [get_ports { sw[1] }];
set_property -dict {PACKAGE_PIN G16 IOSTANDARD LVCMOS33 } 
  [get_ports { sw[2] }];
set_property -dict {PACKAGE_PIN H14 IOSTANDARD LVCMOS33 } 
  [get_ports { sw[3] }];
set_property -dict {PACKAGE_PIN E16 IOSTANDARD LVCMOS33 } 
  [get_ports { sw[4] }];
set_property -dict {PACKAGE_PIN F13 IOSTANDARD LVCMOS33 } 
  [get_ports { sw[5] }];
set_property -dict {PACKAGE_PIN G13 IOSTANDARD LVCMOS33 } 
  [get_ports { sw[6] }];
set_property -dict {PACKAGE_PIN H16 IOSTANDARD LVCMOS33 } 
  [get_ports { sw[7] }];
\end{lstlisting}
效果如下:\\
\includegraphics[width = .9\textwidth]{problem2.jpg}\\
{\bf3).问题三:}利用两个计数器,100hz用于对数码管扫描,10hz用于0.1s的精确度计时;\\
verilog代码如下:
\begin{lstlisting}
  module problem3(
    input rst,
    input clk,
    output reg [3:0] D,
    output reg [2:0] AN
        );
    reg [15:0] sw;
    reg [31:0] cnt_100;
    reg [31:0] cnt_10;
    
    always@(posedge clk)
    begin
        if(cnt_100 >= 1000000)
            cnt_100 <= 0;
         else
            cnt_100 <= cnt_100 + 1;
    end
    
    always@(posedge clk)
    begin
        if(cnt_100 == 0)
        begin
            if(AN == 3'b011)
                AN <= 3'b000;
            else    
                AN <= AN + 3'b001;
        end 
    end
    
    always@(posedge clk)
    begin
        if(cnt_10 >= 10000000)
            cnt_10 <= 0;
         else
            cnt_10 <= cnt_10 + 1;
    end
    
    always@(posedge clk)
    begin 
        if(cnt_10 == 0)
        begin
            if(rst == 1)
                sw = 16'b0001001000110100;
            else
                sw <= sw + 16'b0000000000000001;
            
            if(sw[3:0] == 4'b1010)
            begin
                sw[7:4] <= sw[7:4] + 4'b0001;
                sw[3:0] <= 4'b0000;
            end
            if(sw[7:4] == 4'b1010)
            begin
                sw[11:8] <= sw[11:8] + 4'b0001;
                sw[7:4] <= 4'b0000;
            end     
            if(sw[11:8] == 4'b0110)
            begin
                sw[15:12] <= sw[15:12] + 4'b0001;               
                sw[11:8] <= 4'b0000;               
            end      
            if(sw[15:12] == 4'b1010)
            begin
                sw[15:12] <= 4'b0000;
            end
        end
    end
    
    always@(posedge clk)
    begin
            if(AN == 3'b000)
                D <= sw[3:0];
            else if(AN == 3'b001)
                D <= sw[7:4];
            else if(AN == 3'b010)
                D <= sw[11:8];
            else 
                D <= sw[15:12];
    end
    endmodule
\end{lstlisting}
约束文件如下:
\begin{lstlisting}
set_property  -dict {PACKAGE_PIN E3 IOSTANDARD LVCMOS33 } 
  [get_ports { clk }];
  
set_property -dict {PACKAGE_PIN B18 IOSTANDARD LVCMOS33 } 
  [get_ports { rst }];

set_property -dict {PACKAGE_PIN A14 IOSTANDARD LVCMOS33 } 
  [get_ports { D[0] }];
set_property -dict {PACKAGE_PIN A13 IOSTANDARD LVCMOS33 } 
  [get_ports { D[1] }];
set_property -dict {PACKAGE_PIN A16 IOSTANDARD LVCMOS33 } 
  [get_ports { D[2] }];
set_property -dict {PACKAGE_PIN A15 IOSTANDARD LVCMOS33 } 
  [get_ports { D[3] }];
set_property -dict {PACKAGE_PIN B17 IOSTANDARD LVCMOS33 } 
  [get_ports { AN[0] }];
set_property -dict {PACKAGE_PIN B16 IOSTANDARD LVCMOS33 }
  [get_ports { AN[1] }];
set_property -dict {PACKAGE_PIN A18 IOSTANDARD LVCMOS33 }
  [get_ports { AN[2] }];
\end{lstlisting}
效果如下:\\
\includegraphics[width = .9\textwidth]{problem3.jpg}\\
以及复位时如下:\\
\includegraphics[width = .9\textwidth]{problem3_1.jpg}\\
{\bf5.总结与思考:}\\
{\bf1).收获:}通过本次实验对于FPGA有了相当了解,同时对于Vivado软件的使用了有了更为深入的了解,学会FPGA开发的各环节以及IP核;\\
{\bf2).评价:}实验内容相对而言较难,对于设计以及软件使用难度都有了较大的提升;\\
{\bf3).建议:}实验内容设置合理,无较大建议;
\end{document}

