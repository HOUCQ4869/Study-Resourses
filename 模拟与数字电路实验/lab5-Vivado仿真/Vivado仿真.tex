\documentclass[12pt]{ctexart}
\usepackage{graphicx}
\usepackage{fancyhdr}
\usepackage{amsmath}
\usepackage{geometry}
\geometry{papersize={21cm,29.7cm}}
\geometry{left=2cm,right=2cm,top=3cm,bottom=2cm}
\pagestyle{fancy}
\lhead{姓名:侯超群}
\chead{\today}
\rhead{学号:PB21111618}
\lfoot{}
\cfoot{\thepage}
\rfoot{}
\renewcommand{\headrulewidth}{0.4pt}
\renewcommand{\headwidth}{\textwidth}
\renewcommand{\footrulewidth}{0pt}
\begin{document}
\section*{模拟与数字电路实验报告}
\noindent{\bf1.实验题目:}实验05 使用 Vivado 进行仿真 \\
{\bf2.实验目的:}熟悉 Vivado 软件的下载、安装及使用,学习使用 Verilog 编写仿真文件,学习使用 Verilog 进行仿真,查看并分析波形文件;\\
{\bf3.实验环境:}CPU:AMD Ryzen 5 5600H with Radeon Graphics 3.30GHz;内存:16GB;操作系统:Windows 10;软件平台:Logisim;\\
{\bf4.实验练习:}\\
{\bf1).问题一:}\\
根据题目中波形编写Verilog仿真文件如下(默认时间单位为ns):\\
\`timescale 1ns / 1ps\\
module problem1();\\
reg a, b;\\
initial\\
begin\\
\hspace*{0.5cm}a = 1;\\
\hspace*{0.5cm}\#200 a = 0;\\
\hspace*{0.5cm}\#200 \$stop;\\
end\\
initial\\
begin\\
\hspace*{0.5cm}b = 0;\\
\hspace*{0.5cm}\#100 b = 1;\\
\hspace*{0.5cm}\#175 b = 0;\\
\hspace*{0.5cm}\#75 b = 1;\\
\hspace*{0.5cm}\#50 \$stop;\\
end\\
endmodule\\
\includegraphics[width = .9\textwidth]{problem1仿真.jpg}\\
{\bf2).问题二:}根据波形编写仿真文件如下:\\
\`timescale 1ns / 1ps\\
module problem2();\\
reg CLK, RST\_N, D;\\
initial CLK = 0;\\
always  \#5 CLK = $\sim$ CLK;\\
initial \\
begin\\
\hspace*{0.5cm}RST\_N = 0;\\
\hspace*{0.5cm}\#27.5 RST\_N = 1;\\
end\\
initial\\
begin\\
\hspace*{0.5cm}D = 0;\\
\hspace*{0.5cm}\#12.5 D = 1;\\
\hspace*{0.5cm}\#25 D = 0;\\
\hspace*{0.5cm}\#17.5 \$stop;\\
end\\
endmodule\\
生成如下:\\
\includegraphics[width = .9\textwidth]{problem2.jpg}\\
{\bf3).问题三:}根据问题二的波形作为输入,编写代码如下:(设置了D的波形在12.5以及37.5变化,RST\_N在27.5处变化)\\
\`timescale 1ns / 1ps\\
module problem3(\\
input clk, rst\_n, d,\\
output reg q\\
\hspace*{0.5cm});\\
always@(posedge clk)\\
begin\\
\hspace*{0.5cm}if(rst\_n == 0)\\
\hspace*{1cm}q <=  1'b0;\\
\hspace*{0.5cm}else \\
\hspace*{1cm}q <= d;\\
end\\
endmodule\\
仿真测试文件如下:\\
\`timescale 1ns / 1ps\\
module test\_bench();\\
reg clk, rst\_n, d;\\
wire q;\\
problem3 problem3(.clk(clk), .rst\_n(rst\_n), .d(d), .q(q));\\
initial clk = 0;\\
always \#5 clk = $\sim$ clk;\\
initial\\
begin\\
\hspace*{0.5cm}rst\_n = 0;\\
\hspace*{0.5cm}\#27.5 rst\_n = 1; \\
end\\
initial\\
begin\\
\hspace*{0.5cm}d = 0;\\
\hspace*{0.5cm}\#12.5 d = 1;\\
\hspace*{0.5cm}\#25 d = 0;\\
\hspace*{0.5cm}\#17.5 \$finish;\\
end\\
endmodule\\
仿真波形如下:\\
\includegraphics[width = .9\textwidth]{problem3.jpg}\\
{\bf4).问题四:}3-8译码器代码如下:\\
module problem4(\\
\hspace*{0.5cm}input [2:0] A,\\
\hspace*{0.5cm}input E,\\
\hspace*{0.5cm}output reg [7:0] Y\\
);\\
integer k;\\
always@(*)\\
\hspace*{0.5cm}for(k = 0; k <= 7; k = k + 1)\\
\hspace*{0.5cm}begin\\
\hspace*{1cm}if((E == 1) \&\& (A == k))\\
\hspace*{1.5cm}Y[k] = 1;\\
\hspace*{1cm}else\\
\hspace*{1.5cm}Y[k] = 0;\\
\hspace*{0.5cm}end\\
endmodule\\
仿真测试文件如下:(其中对于使能端设置了在5,15,25时的电平翻转,测试使能端作用,A端口每隔10个时间单位变化一次)\\
\`timescale 1ns / 1ps\\
module test\_bench();\\
reg [2:0] A;\\
reg E;\\
wire [7:0]  Y;\\
problem4 problem4(.A(A), .E(E), .Y(Y));\\
initial\\
begin\\
\hspace*{0.5cm}E = 0;\\
\hspace*{0.5cm}\#5 E = 1;\\
\hspace*{0.5cm}\#10 E = 0;\\
\hspace*{0.5cm}\#10 E = 1;\\
end\\
initial\\
begin\\
\hspace*{0.5cm}A[  2 ]  = 0; A [  1 ] = 0; A[  0]  = 0; \\
\hspace*{0.5cm}\#10  A[  2 ] = 0; A[  1 ] = 0; A [  0 ]  = 1; \\
\hspace*{0.5cm}\#10  A [  2 ] = 0; A [  1 ] = 1; A [  0 ]  = 0; \\
\hspace*{0.5cm}\#10  A [  2 ]= 0; A  [  1] = 1; A [  0]  = 1; \\
\hspace*{0.5cm}\#10  A [  2 ]= 1; A [  1 ]  = 0; A [  0 ]  = 0; \\
\hspace*{0.5cm}\#10  A[  2 ]= 1; A[  1 ]  = 0; A [  0 ]  = 1; \\
\hspace*{0.5cm}\#10  A [  2 ]= 1; A[  1 ]  = 1; A [  0 ]  = 0; \\
\hspace*{0.5cm}\#10  A [  2 ]= 1; A [  1 ]  = 1; A [  0]  = 1; \\
\hspace*{0.5cm}\#10 \$finish;\\
end\\
endmodule\\
\includegraphics[width = .9\textwidth]{problem4.jpg}\\
{\bf5.总结与思考:}\\
{\bf1).收获:}本次实验对于Vivado软件的使用了有了更为深入的了解,学会了仿真文件的编写与测试,以及对波形文件的分析;\\
{\bf2).评价:}实验作为Vivado软件使用的基本内容以及对仿真文件的介绍,其内容相对而言不是较难,设置合理;\\
{\bf3).建议:}实验内容设置合理,无较大建议;
\end{document}

