\documentclass[12pt]{ctexart}
\usepackage{graphicx}
\usepackage{fancyhdr}
\usepackage{amsmath}
\usepackage{geometry}
\geometry{papersize={21cm,29.7cm}}
\geometry{left=2cm,right=2cm,top=3cm,bottom=2cm}
\pagestyle{fancy}
\lhead{姓名:侯超群}
\chead{\today}
\rhead{学号:PB21111618}
\lfoot{}
\cfoot{\thepage}
\rfoot{}
\renewcommand{\headrulewidth}{0.4pt}
\renewcommand{\headwidth}{\textwidth}
\renewcommand{\footrulewidth}{0pt}
\begin{document}
\section*{模拟与数字电路实验报告}
\noindent{\bf1.实验题目:}实验05 FPGA原理及Vivado综合 \\
{\bf2.实验目的:}了解 FPGA 工作原理,了解 Verilog 文件和约束文件在 FPGA 开发中的作用,学会使用 Vivado 进行 FPGA 开发的完整流程\\
{\bf3.实验平台:}Vivado及Logisim软件;\\
{\bf4.实验练习:}\\
{\bf1).问题一:}通过实验中给出的可编程逻辑单元,交叉互联矩阵及IOB电路图,实现题目中所给出的代码\\
由于输入端为a,以及常量1,实现其异或关系,并运用时序逻辑电路,电路图及相应配置如下:\\
\includegraphics[width = .9\textwidth]{problem1.jpg}\\
{\bf2).问题二:}根据实验中所给出的XDC约束文件进行相应修改,将输入处或输出处管脚对应的前后调换即可,相应约束文件代码如下:\\
\includegraphics[width = .9\textwidth]{problem2_code.jpg}\\
烧写后在实验平台显示截图如下:\\
\includegraphics[width = .9\textwidth]{problem2.jpg}\\
{\bf3).问题三:}设计30位计数器,以及32位,其原理类似,下面以32位为例,相应代码及约束文件如下:\\
\includegraphics[width = .9\textwidth]{problem3_32_code.jpg}\\\\
\includegraphics[width = .9\textwidth]{problem3_32_xpc.jpg}\\
其在实验平台显示截图如下:\\
\includegraphics[width = .9\textwidth]{problem3.jpg}\\
通过对比不难发现,相对而言30位计数器变化的更快,其原因在于位数较低,变化所需时间周期更短,为$10^{-8}\times  2^{22}$s;\\
{\bf5.总结与思考:}\\
{\bf1).收获:}通过本次实验对于FPGA有了相当了解,同时对于Vivado软件的使用了有了更为深入的了解,学会了约束文件的编写与测试;\\
{\bf2).评价:}实验内容相对而言不是较难,设置合理;\\
{\bf3).建议:}实验内容设置合理,无较大建议;
\end{document}

