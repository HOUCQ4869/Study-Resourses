\documentclass[12pt]{ctexart}
\usepackage{graphicx}
\usepackage{fancyhdr}
\usepackage{amsmath}
\usepackage{geometry}
\usepackage{xcolor}
\usepackage{listings}

\lstset{
 columns=fixed,       
 numbers=left,                                        % 在左侧显示行号
 numberstyle=\tiny\color{gray},                       % 设定行号格式
 frame=none,                                          % 不显示背景边框
 backgroundcolor=\color[RGB]{245,245,244},            % 设定背景颜色
 keywordstyle=\color[RGB]{40,40,255},                 % 设定关键字颜色
 numberstyle=\footnotesize\color{darkgray},           
 commentstyle=\it\color[RGB]{0,96,96},                % 设置代码注释的格式
 stringstyle=\rmfamily\slshape\color[RGB]{128,0,0},   % 设置字符串格式
 showstringspaces=false,                              % 不显示字符串中的空格
 language=c,                                        % 设置语言
}

\geometry{papersize={21cm,29.7cm}}
\geometry{left=2cm,right=2cm,top=3cm,bottom=2cm}
\pagestyle{fancy}
\lhead{姓名:侯超群}
\chead{\today}
\rhead{学号:PB21111618}
\lfoot{}
\cfoot{\thepage}
\rfoot{}
\renewcommand{\headrulewidth}{0.4pt}
\renewcommand{\headwidth}{\textwidth}
\renewcommand{\footrulewidth}{0pt}
\begin{document}
\section*{模拟与数字电路实验报告}
\noindent{\bf1.实验题目:}实验08 信号处理及有限状态机 \\
{\bf2.实验目的:}进一步熟悉 FPGA 开发的整体流程,掌握几种常见的信号处理技巧,掌握有限状态机的设计方法,能够使用有限状态机设计功能电路\\
{\bf3.实验平台:}Vivado及Logisim软件;\\
{\bf4.实验练习:}\\
{\bf1).问题一:}将实验手册中的代码改成三段式有限状态机形式如下:\\
\begin{lstlisting}
module problem1(
input clk,
input rst,
output led
);
parameter C_0 = 2'b00;
parameter C_1 = 2'b01;
parameter C_2 = 2'b10;
parameter C_3 = 2'b11;
reg [1:0] curr_state;
reg [1:0] next_state;
//第一部分
always@(*)
begin
    case(curr_state)
        C_0 : next_state = C_1;
        C_1 : next_state = C_2;
        C_2 : next_state = C_3;
        C_3 : next_state = C_0;
    endcase
end
//第二部分
always@(posedge clk or posedge rst)
begin
    if(rst)
        curr_state <= C_0;
    else 
        curr_state <= next_state;
end
//第三部分
assign led = (curr_state == 2'b11) ? 1'b1 : 1'b0;
endmodule    
\end{lstlisting}
{\bf2).问题二:}在logisim中设计4bit位宽的计数器电路,由于根据sw电平变化进行计数,需要对信号进行取边沿,对实验手册中的设计进行稍加修改,使其在电平翻转均能产生脉冲;
截图如下:\\
\includegraphics[width = .9\textwidth]{problem2.jpg}\\
其中调用4bit寄存器模块,如下:\\
\includegraphics[width = .9\textwidth]{problem2-.jpg}\\
{\bf3).问题三:}设计8位十六进制计数器,主要对于开关按钮进行取边沿操作,使其产生计数,具体代码如下:\\
\begin{lstlisting}
module problem3(
input clk, button, rst, sw,
output reg [3:0]D,
output reg [2:0]AN
);
//一个时钟周期
reg button_r1, button_r2;
wire button_edge;
always@(posedge clk)
    button_r1 <= button;
always@(posedge clk)
    button_r2 <= button_r1;
assign button_edge = button_r1 & (~button_r2);
//扫描
reg [15:0]cnt;
reg [7:0]led;
always@(posedge clk)
begin
    if(cnt >= 10000)
        cnt <= 0;
    else 
        cnt <= cnt + 1;
end

always@(posedge clk)
begin
    if(cnt == 0)
    begin
        if(AN == 3'b000)
            AN = 3'b001;
        else
            AN = 3'b000;
    end
end

always@(posedge clk)
begin
    if(AN == 3'b000)
        D <= led[3:0];
    else 
        D <= led[7:4];
end
//计数
always@(posedge clk)
begin
    if(rst == 1)
        led = 8'b00011111;
    if(button_edge)
    begin
        if(sw)
            led = led + 8'b00000001;
        else
            led = led - 8'b00000001;
    end
end
endmodule
\end{lstlisting}
复位时其在实验平台显示截图如下:\\
\includegraphics[width = .9\textwidth]{problem1.jpg}\\
{\bf4).问题四:}使用有限状态机设计该序列监测电路,共5个状态,可将其化简为4种。并通过对开关信号的取边沿判断有限状态机的下一步走向,最后赋给相关管脚;\\
代码如下:\\
\begin{lstlisting}
  module problem4(
    input sw, clk, button,
    output reg [3:0]D,
    output reg [2:0]AN
        );
    reg [3:0]count;
    reg [15:0]num;
    //一个时钟周期
    reg button_r1, button_r2;
    wire button_edge;
    always@(posedge clk)
        button_r1 <= button;
    always@(posedge clk)
        button_r2 <= button_r1;
    assign button_edge = button_r1 & (~button_r2);
    //扫描
    reg [15:0]cnt;
    always@(posedge clk)
    begin
        if(cnt >= 2500)
            cnt <= 0;
        else    
            cnt <= cnt + 1;
    end
    always@(posedge clk)
    begin
        if(cnt == 0)
        begin
            if(AN == 3'b101)
                AN <= 3'b000;
            else 
                AN <= AN + 3'b001;
        end
    end
    always@(posedge clk)
    begin
        if(AN == 3'b000)
            D <= num[3:0];
        else if(AN == 3'b001)
            D <= count;
        else if(AN == 3'b010)
            D <= num[3:0];
        else if(AN == 3'b011)
            D <= num[7:4];
        else if(AN == 3'b100)
            D <= num[11:8];
        else 
            D <= num[15:12];
    end
    //状态机
    parameter C_0 = 2'b00;
    parameter C_1 = 2'b01;
    parameter C_11 = 2'b10;
    parameter C_110 = 2'b11;
    reg [1:0] curr_state;
    reg [1:0] next_state;
    always@(*)
    begin
        if(sw)
        begin   
            case(curr_state)
                C_0: next_state = C_1;
                C_1: next_state = C_11;
                C_11: next_state = C_11;
                C_110: next_state = C_1;
                default: next_state = C_0;
            endcase
        end
        else
        begin
            case(curr_state)
                C_0: next_state = C_0;
                C_1: next_state = C_0;
                C_11: next_state = C_110;
                C_110: next_state = C_0;
                default: next_state = C_0;
            endcase
        end
    end
    
    always@(posedge clk)
    begin
        if(button_edge)
        begin
            if(curr_state == C_110 && next_state == C_0)
                count = count + 1;
            curr_state <= next_state;
            //根据状态变化
            num[15:12] <= num[11:8];
            num[11:8] <= num[7:4];
            num[7:4] <= num[3:0];
            num[3:0] <= {3'b000,sw};                
        end
    end
    
    endmodule
\end{lstlisting}
当输入0011001110011时其在实验平台显示截图如下:分别显示最近输入的4个数值,检测到目标序列的个数,当前输入的数值\\
\includegraphics[width = .9\textwidth]{problem3.jpg}\\
{\bf5.总结与思考:}\\
{\bf1).收获:}通过本次实验对于FPGA有了相当了解,同时对于Vivado软件的使用了有了更为深入的了解,学会了信号处理以及有限状态机的使用;\\
{\bf2).评价:}实验内容由于有前面实验的铺垫相对而言不是较难,设置合理;\\
{\bf3).建议:}实验内容设置合理,无较大建议;
\end{document}

