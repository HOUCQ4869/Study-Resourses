\documentclass[12pt]{ctexart}
\usepackage{graphicx}
\usepackage{fancyhdr}
\usepackage{amsmath}
\usepackage{geometry}
\geometry{papersize={21cm,29.7cm}}
\geometry{left=2cm,right=2cm,top=3cm,bottom=2cm}
\pagestyle{fancy}
\lhead{姓名:侯超群}
\chead{\today}
\rhead{学号:PB21111618}
\lfoot{}
\cfoot{\thepage}
\rfoot{}
\renewcommand{\headrulewidth}{0.4pt}
\renewcommand{\headwidth}{\textwidth}
\renewcommand{\footrulewidth}{0pt}
\begin{document}
\section*{实验报告}
\noindent{\bf1.Report name:}Lab03\\
{\bf2.Purpose:}\\
给定一个字符串及其长度,计算出该字符串最长的重复子字符串的长度;其中x3100存储N,即字符串长度,
紧接着每个字符存储在从地址x3101开始的连续存储位置中;\\
要求将结果储存到x3050中,程序从x3000开始;\\
{\bf3.Principles:}\\
1)首先用R1指向初始位置,R3,R4用于存储该位置的字符值以及相邻一位的值;R1每次循环不断向后移动,比较R3,R4所存储的字符值是否相等;\\
2)取反加一对R3,R4进行减法运算,相等则跳转至计数语句COUNT,否则对记录计数的R7初始化为1,重新开始比较;\\
3)计数语句实现对当前连续相等字符数R7的更新,同时与最大重复字符串长度R2比较,更新R2;\\
4)R0作为循环截止条件,不管是否进入计数,都根据此时R0的值判断是否跳出循环;\\
{\bf4.Procedure:}\\
.ORIG x3000\\
\hspace*{1.5cm}LDI R0, NUM\\
\hspace*{1.5cm}LD R1, DATA\\
\\
\hspace*{1.5cm}ADD R2, R2, \#1\\
\hspace*{1.5cm}ADD R7, R7, \#1\\
\hspace*{1.5cm};初始化R2, R7计数为1\\
AGAIN   ADD R0, R0, \#-1\\
\hspace*{1.5cm}BRz OVER\\
\hspace*{1.5cm}LDR R3, R1, \#0\\
\hspace*{1.5cm}LDR R4, R1, \#1\\
\hspace*{1.5cm}ADD R1, R1, \#1\\
\hspace*{1.5cm};得到相邻的字符,并将指针向后推移一位\\
\hspace*{1.5cm}NOT R5, R3\\
\hspace*{1.5cm}ADD R5, R5 \#1\\
\hspace*{1.5cm}ADD R6, R5, R4\\
\hspace*{1.5cm};减法计算是否相等,则跳转到COUNT\\
\hspace*{1.5cm}BRz COUNT\\
\hspace*{1.5cm}AND R7, R7, \#0\\
\hspace*{1.5cm}ADD R7, R7, \#1\\
\hspace*{1.5cm};重新赋值计数\\
\hspace*{1.5cm}BRnzp AGAIN\\
\\
COUNT   ADD R7, R7, \#1\\
\hspace*{1.5cm}NOT R3, R2\\
\hspace*{1.5cm}ADD R3, R3, \#1\\
\hspace*{1.5cm}ADD R4, R7, R3\\
\hspace*{1.5cm}BRnz AGAIN\\
\hspace*{1.5cm};比较此时相等长度与最大长度R2的大小\\
\hspace*{1.5cm}ADD R2, R7 \#0\\
\hspace*{1.5cm}BRnzp AGAIN\\
\\
OVER    STI R2, RESULT\\
\hspace*{1.5cm}HALT\\
\\
RESULT .FILL x3050\\
NUM .FILL x3100\\
DATA .FILL x3101\\
\\
.END\\
{\bf5.Result of test:}\\
根据自测网站,评测结果如下:\\
\includegraphics[width = .9\textwidth]{评测.jpg}\\
\end{document}

