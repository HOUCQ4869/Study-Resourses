\documentclass[12pt]{ctexart}
\usepackage{graphicx}
\usepackage{fancyhdr}
\usepackage{amsmath}
\usepackage{geometry}
\geometry{papersize={21cm,29.7cm}}
\geometry{left=2cm,right=2cm,top=3cm,bottom=2cm}
\pagestyle{fancy}
\lhead{姓名:侯超群}
\chead{\today}
\rhead{学号:PB21111618}
\lfoot{}
\cfoot{\thepage}
\rfoot{}
\renewcommand{\headrulewidth}{0.4pt}
\renewcommand{\headwidth}{\textwidth}
\renewcommand{\footrulewidth}{0pt}
\begin{document}
\section*{实验报告}
\noindent{\bf1.实验题目:}计算系统概论Lab01\\
{\bf2.实验目的:}\\
利用LC-3语言编写程序,计算给定数字A的较低B位中有多少个1,
并输出储存在内存中;\\
{\bf1).实验要求:}实验中要求程序从内存位置x3000开始,A和B的值分别在x3100和x3101中,
并假设A值是一个从0x0001到0x7FFFF之间的正数,要求将输出储存在x3102中;\\
注意B的范围为0到16;\\
{\bf3.实验原理:}\\
首先以寄存器R1,R2储存A,B的值(注意当B的值为0时,可根据R2储存的值设置跳转语句),寄存器R3赋值为1,用于位运算,用寄存器R4表示R1与R3进行与运算后的结果,
每次循环时R3的值乘2,等同于从A最低位逐次判断是否为1,每次循环R2减一,直至等于0时比较结束,
同时以寄存器R5记录1的个数,根据与运算操作后的R4的值设置一个跳转语句,为0时跳转,否则执行R5=R5+1;
最后还有一个跳转语句用来实现循环,根据R2=R2-1的值,若为正,则跳回,否则继续执行写入内存的操作,结束程序;\\
{\bf4.实验步骤:}\\
第一行:起始地址\\
x3000:LD指令,将x3100处的值赋给R1,R1<-M[x3100];\\
x3001:LD指令,将R3赋值为1,在x300d处设置内容为1,R3<-M[x300d],即R3<-1;\\
x3002:ADD指令,将R5赋值为0,用R3-1赋给R5,即R5=R3-1=0;\\
x3003:LD指令,将x3101处的值赋给R2,R2<-M[x3101];\\
x3004:BR指令,根据R2的值是否为零,为零直接跳转到x300b,将R5的值存到x3102中\\
x3005:AND指令,对R1与R3进行与运算,即从最低位逐位比较是否为0,将结果存至R4;\\
x3006:BR指令,根据上一条指令中R4的结果是否为1,判断是否跳转,若该位为1,则执行x3007语句,否则跳转至x3008;\\
x3007:ADD指令,若R4结果为1,执行R5=R5+1,即统计结果中1的个数增加;\\
x3008:ADD指令,R3=R3+R3,每次循环R3乘2,便于逐位进行与运算;\\
x3009:ADD指令,R2=R2-1,用于记录比较的位数,当R2为0时,即比较完A的前B位;\\
x300a:BR指令,根据上一条指令中R2的结果是否为正,若为正则跳回x3005,继续循环,进行A的逐位比较,否则结束比较进行下一步;\\
x300b:ST指令:此时以确定A的前B位中1的个数,将R5的值储存到x3102处即可;\\
x300c:TRAP指令,结束程序;\\
x300d:用于储存赋给R2的值;\\
{\bf5.实验结果:}具体程序如下:\\
0011 000 000000000    ;起始地址 \\
0010 001 011111111    ;x3000 R1<-M[x3100]  \\
0010 011 000001010    ;x3001 R3<-M[x300d];R3<-1\\
0001 101 011 1 11111  ;x3002 R5<-R3-1;R5<-0\\
0010 010 011111101    ;x3003 R2<-M[x3101]\\
0000 010 000000110    ;x3004 BRZ 010 X300b\\
0101 100 001 0 00 011 ;x3005 AND R4 R1 R3\\
0000 010 000000001    ;x3006 BRZ 010 x3008\\
0001 101 101 1 00001  ;x3007 R5<-R5+1\\
0001 011 011 0 00 011 ;x3008 R3<-R3+R3\\
0001 010 010 1 11111  ;x3009 R2<-R2-1 \\
0000 001 111111010    ;x300a BRP 001 x3005\\
0011 101 011110110    ;x300b M[x3102]<-R5\\
1111 0000 00011001    ;x300c TRAP 25\\
0000 0000 0000 0001   ;x300d 1\\
\includegraphics[width = .9\textwidth]{代码.jpg}\\
根据自测网站,适当添加数据测试,评测结果如下:\\
\includegraphics[width = .9\textwidth]{评测.jpg}\\
\end{document}

