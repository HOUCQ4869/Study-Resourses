\documentclass[12pt]{ctexart}
\usepackage{graphicx}
\usepackage{fancyhdr}
\usepackage{amsmath}
\usepackage{geometry}
\geometry{papersize={21cm,29.7cm}}
\geometry{left=2cm,right=2cm,top=3cm,bottom=2cm}
\pagestyle{fancy}
\lhead{姓名:侯超群}
\chead{\today}
\rhead{学号:PB21111618}
\lfoot{}
\cfoot{\thepage}
\rfoot{}
\renewcommand{\headrulewidth}{0.4pt}
\renewcommand{\headwidth}{\textwidth}
\renewcommand{\footrulewidth}{0pt}
\begin{document}
\section*{实验报告}
\noindent{\bf1.Report name:}Lab02\\
{\bf2.Purpose:}\\
根据题中所给的p,q,N的值(分别位于x3100,x3101,x3102),计算F(N),并将其存储在x3103处;\\
其中$F(N)=F(N-2)\%p+F(N-1)\%q,(2\le N\le 1024),p=2^{k} (2\le k\le 10),10\le q\le 1024$\\
{\bf3.Principles:}\\
1):在设计解法时,要求计算第N个,因此要以一个寄存器存储N的值,并每次循环时递减;\\
2):对于该循环过程中对于取模的操作,可以嵌套两个循环,分别计算取模p,q后的值,在该循环中,寄存器存储F(N-1)或F(N-2)的值,
每次减去p或q,当减至负数时跳出,再加上p或q,得到取模后值;\\
3):对于减法操作,可以以一个新寄存器存储p或q,取反加一后进行ADD运算即可;\\
{\bf4.Procedure:}\\
\hspace*{1.5cm}.ORIG x3000\\
\hspace*{1.5cm}LDI  R0, PPTR\\
\hspace*{1.5cm}LDI  R1, QPTR\\
\hspace*{1.5cm}LDI  R2, NPTR\\
\hspace*{1.5cm}ADD R5, R5, \#1\\
\hspace*{1.5cm}ADD R6, R6, \#1\\
\hspace*{1.5cm}ADD R2, R2, \#-2\\
\hspace*{1.5cm};赋初值,N-2便于跳出\\
\hspace*{1.5cm}BRn STORE\\
AGAIN   ADD R3, R5, \#0\\
\hspace*{1.5cm}ADD R4, R6, \#0\\
\hspace*{1.5cm}ADD R5, R4, \#0\\
\hspace*{1.5cm};R3,R4储存F(N-1),F(N-2),R5记录便于下一个循环使用\\
\hspace*{1.5cm}NOT R7, R0\\
\hspace*{1.5cm}ADD R7, R7, \#1\\
SUBA  ADD R3, R3, R7\\
\hspace*{1.5cm}BRzp SUBA\\
\hspace*{1.5cm};对p取模\\
\hspace*{1.5cm}NOT R7, R1\\
\hspace*{1.5cm}ADD R7, R7, \#1\\
SUBB  ADD R4, R4, R7\\
\hspace*{1.5cm}BRzp SUBB\\
\hspace*{1.5cm};对q取模\\
\hspace*{1.5cm}ADD R3, R3, R0\\
\hspace*{1.5cm}ADD R4, R4, R1\\
\hspace*{1.5cm};取模后的负值再加上原值\\
\hspace*{1.5cm}ADD R6, R3, R4\\
\hspace*{1.5cm}ADD R2, R2, \#-1\\
\hspace*{1.5cm}BRzp AGAIN\\
\hspace*{1.5cm};根据N值判断是否循环\\
STORE STI R6, RESULT\\
\hspace*{1.5cm}TRAP x25\\
\hspace*{1.5cm};HALT\\
\hspace*{1.5cm}PPTR    .FILL   x3100\\
\hspace*{1.5cm}QPTR    .FILL   x3101\\
\hspace*{1.5cm}NPTR    .FILL   x3102\\
\hspace*{1.5cm}RESULT     .FILL   x3103\\
\hspace*{1.5cm}.END\\
{\bf5.Result of test:}\\
根据自测网站,评测结果如下:\\
\includegraphics[width = .9\textwidth]{评测.jpg}\\
{\bf6.Idea:}\\
由于p为2的k次方的形式,在二进制中取模对其进行位操作即可,如1010010mod10000,则比q低位处0010即为取模后余数;\\
可根据此来对对q取模这一循环结构进行优化;
\end{document}

